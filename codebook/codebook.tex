\documentclass[a4paper,10pt,twocolumn,oneside]{article}
\setlength{\columnsep}{10pt}                                                                    %兩欄模式的間距
\setlength{\columnseprule}{0pt}                                                                %兩欄模式間格線粗細

\usepackage{amsthm}								%定義,例題
\usepackage{amssymb}
%\usepackage[margin=2cm]{geometry}
\usepackage{fontspec}								%設定字體
\usepackage{color}
\usepackage[x11names]{xcolor}
\usepackage{listings}								%顯示code用的
\usepackage[Glenn]{fncychap}						%排版,頁面模板
\usepackage{fancyhdr}								%設定頁首頁尾
\usepackage{graphicx}								%Graphic
\usepackage{enumerate}
\usepackage{multicol}
\usepackage{titlesec}
\usepackage{amsmath}
% \usepackage[CheckSingle, CJKmath]{xeCJK}
% \usepackage{CJKulem}

%\usepackage[T1]{fontenc}
\usepackage{amsmath, courier, listings, fancyhdr, graphicx}
\topmargin=0pt
\headsep=5pt
\textheight=780pt
\footskip=0pt
\voffset=-40pt
\textwidth=545pt
\marginparsep=0pt
\marginparwidth=0pt
\marginparpush=0pt
\oddsidemargin=0pt
\evensidemargin=0pt
\hoffset=-42pt

\titlespacing\subsection{0pt}{5pt plus 4pt minus 2pt}{0pt plus 2pt minus 2pt}


%\renewcommand\listfigurename{圖目錄}
%\renewcommand\listtablename{表目錄} 

%%%%%%%%%%%%%%%%%%%%%%%%%%%%%

\setmainfont{Consolas}				%主要字型
%\setmonofont{Monaco}				%主要字型
\setmonofont{Consolas}
% \setCJKmainfont{Noto Sans CJK TC}
% \setCJKmainfont{Consolas}			%中文字型
%\setmainfont{sourcecodepro}
\XeTeXlinebreaklocale "zh"						%中文自動換行
\XeTeXlinebreakskip = 0pt plus 1pt				%設定段落之間的距離
\setcounter{secnumdepth}{3}						%目錄顯示第三層

%%%%%%%%%%%%%%%%%%%%%%%%%%%%%
\makeatletter
\lst@CCPutMacro\lst@ProcessOther {"2D}{\lst@ttfamily{-{}}{-{}}}
\@empty\z@\@empty
\makeatother
\lstset{											% Code顯示
language=C++,										% the language of the code
basicstyle=\footnotesize\ttfamily, 						% the size of the fonts that are used for the code
%numbers=left,										% where to put the line-numbers
numberstyle=\footnotesize,						% the size of the fonts that are used for the line-numbers
stepnumber=1,										% the step between two line-numbers. If it's 1, each line  will be numbered
numbersep=5pt,										% how far the line-numbers are from the code
backgroundcolor=\color{white},					% choose the background color. You must add \usepackage{color}
showspaces=false,									% show spaces adding particular underscores
showstringspaces=false,							% underline spaces within strings
showtabs=false,									% show tabs within strings adding particular underscores
frame=false,											% adds a frame around the code
tabsize=2,											% sets default tabsize to 2 spaces
captionpos=b,										% sets the caption-position to bottom
breaklines=true,									% sets automatic line breaking
breakatwhitespace=false,							% sets if automatic breaks should only happen at whitespace
escapeinside={\%*}{*)},							% if you want to add a comment within your code
morekeywords={constexpr},									% if you want to add more keywords to the set
keywordstyle=\bfseries\color{Blue1},
commentstyle=\itshape\color{Red4},
stringstyle=\itshape\color{Green4},
}

%%%%%%%%%%%%%%%%%%%%%%%%%%%%%

\begin{document}
\pagestyle{fancy}
\fancyfoot{}
%\fancyfoot[R]{\includegraphics[width=20pt]{ironwood.jpg}}
\fancyhead[L]{National Taiwan University PCkomachi}
\fancyhead[R]{\thepage}
\renewcommand{\headrulewidth}{0.4pt}
\renewcommand{\contentsname}{Contents} 

\scriptsize
\textbf{
\scriptsize
\begin{multicols}{2}
  \tableofcontents
\end{multicols}
}
%%%%%%%%%%%%%%%%%%%%%%%%%%%%%

%\newpage

\section{Basic}
\inputcode{Template}{Basic/Template.cpp}
\inputcode{Fast IO}{Basic/FastIO.cpp}
\inputcode{vimrc}{Basic/vimrc}

\section{Graph}
\inputcode{2SAT (SCC)}{Graph/2SAT.cpp}
\inputcode{VertexBCC}{Graph/VertexBCC.cpp}
\inputcode{EdgeBCC}{Graph/EdgeBCC.cpp}
\inputcode{Centroid Decomposition}{Graph/CD.cpp}
\inputcode{Count Cycles}{Graph/CountCycles.cpp}
\inputcode{DirectedMST}{Graph/DirectedMST.cpp}
\inputcode{Dominator Tree}{Graph/DominatorTree.cpp}
\inputcode{Heavy Light Decomposition}{Graph/HLD.cpp}
\inputcode{Matroid Intersection}{Graph/MatroidIntersection.cpp}
%\inputcode{SCC}{Graph/SCC.cpp}
%\inputcode{SPFA}{Graph/SPFA.cpp}
\inputcode{Virtual Tree}{Graph/VirtualTree.cpp}
\inputcode{Vizing}{Graph/Vizing.cpp}
\inputcode{Maximum Clique Dynamic}{Graph/MaxCliqueDyn.cpp}

\subsection{Theory}
\begin{footnotesize}
$|$Maximum independent edge set$|=|V|-|$Minimum edge cover$|$\\
$|$Maximum independent set$|=|V|-|$Minimum vertex cover$|$\\
\end{footnotesize}

\section{Data Structure}
%\inputcode{BIT}{Data Structure/BIT.cpp}
%\inputcode{Chtholly Tree}{Data Structure/ChthollyTree.cpp}
\inputcode{LiChao Tree}{DataStructure/LiChao.cpp}
\inputcode{Dynamic Line Hull}{DataStructure/DynamicLineHull.cpp}
\inputcode{Leftist Tree}{DataStructure/LeftistTree.cpp}
\inputcode{Link Cut Tree}{DataStructure/LCT.cpp}
%\inputcode{Persistent Segment Tree}{Data Structure/PersistentSegTree.cpp}
%\inputcode{Rollback DSU}{Data Structure/RollbackDSU.cpp}
%\inputcode{Sparse Table}{DataStructure/SparseTable.cpp}
\inputcode{Splay Tree}{DataStructure/Splay.cpp}
\inputcode{Treap}{DataStructure/Treap.cpp}

\section{Flow/Matching}
\inputcode{Hopcroft Karp}{Flow/HopcroftKarp.cpp}
\inputcode{Dinic}{Flow/Dinic.cpp}
\inputcode{Min Cost Max Flow}{Flow/MCMF.cpp}
\inputcode{Min Cost Circulation}{Flow/MinCostCirculation.cpp}
\inputcode{Kuhn Munkres}{Flow/KM.cpp}
\inputcode{Stoer Wagner (Min-cut)}{Flow/StoerWagner.cpp}
\inputcode{GomoryHu Tree}{Flow/GomoryHu.cpp}
\inputcode{General Graph Matching}{Flow/GeneralMatching.cpp}
\subsection{Flow notes}
%\lstinputlisting{Flow/notes.txt}
\begin{itemize}
	\itemsep-0.3em
	\item Bipartite Matching Restore Answer

	\texttt{runBfs();} Answer is $\{!vis[x] | x \in L\} \cup \{vis[x] | x \in R\}$
	
	\item Bipartite Minimum Weight Vertex Covering

	$S \to \{x | x \in L\}$, cap = weight of vertex $x$\\
	$\{x | x \in L\} \to \{y | y \in R\}$, cap = $\infty$\\
	$\{y | y \in R\} \to T$, cap = weight of vertex $y$
	
	For general version, change Dinic to MCMF and:
	
	$S \to \{x | x \in L\}$, cap = weight of vertex $x$, cost = $0$\\
	$\{x | x \in L\} \to \{y | y \in R\}$, cap = $\infty$, cost = $-w$\\
	$\{y | y \in R\} \to T$, cap = weight of vertex $y$, cost = $0$
	
	\item Useful Lemma

	(Bipartite Maximum Weight Independent Set) + \\
	(Bipartite Minimum Weight Vertex Covering) = weight sum

	\item Min Cut Model

	choose $A$ but not choose $B$ cost $x$: $A \to B$, cap = $x$\\
	choose $A$ cost $x$: $A \to T$, cap = $x$\\
	not choose $A$ cost $x$: $S \to A$, cap = $x$\\
	choose $A$ gain $x$ $\implies$ not choose $A$ cost $x$, $tot += x$\\
	choose $A$ and choose $B$ cost $x$: NO!!!\\
	Bipartite: can flip one side
	
	\item Min Cut Restore Answer
	
	runBfs(); Answer is $\{vis[x] | x \in V\}$
\end{itemize}
%\subsection{More Flow Models}
% \normalsize
\begin{itemize}
    \itemsep-0.3em
    \item Maximum/Minimum flow with lower bound / Circulation problem
    \vspace{-1em}
    \begin{enumerate}
        \itemsep-0.3em
        \item Construct super source $S$ and sink $T$.
        \item For each edge $(x, y, l, u)$, connect $x \rightarrow y$ with capacity $u - l$.
        \item For each vertex $v$, denote by $in(v)$ the difference between the sum of incoming lower bounds and the sum of outgoing lower bounds.
        \item If $in(v) > 0$, connect $S \rightarrow v$ with capacity $in(v)$, otherwise, connect $v \rightarrow T$ with capacity $-in(v)$.
        \begin{itemize}
            \itemsep-0.2em
            \item To maximize, connect $t \rightarrow s$ with capacity $\infty$ (skip this in circulation problem), and let $f$ be the maximum flow from $S$ to $T$. If $f \neq \sum_{v \in V, in(v) > 0}{in(v)}$, there's no solution. Otherwise, the maximum flow from $s$ to $t$ is the answer.
            \item To minimize, let $f$ be the maximum flow from $S$ to $T$. Connect $t \rightarrow s$ with capacity $\infty$ and let the flow from $S$ to $T$ be $f^\prime$. If $f + f^\prime \neq \sum_{v \in V, in(v) > 0}{in(v)}$, there's no solution. Otherwise, $f^\prime$ is the answer.
        \end{itemize}
        \item The solution of each edge $e$ is $l_e + f_e$, where $f_e$ corresponds to the flow of edge $e$ on the graph.
    \end{enumerate}
    \item Construct minimum vertex cover from maximum matching $M$ on bipartite graph $(X, Y)$
    \vspace{-1em}
    \begin{enumerate}
        \itemsep-0.3em
        \item Redirect every edge: $y \rightarrow x$ if $(x, y) \in M$, $x \rightarrow y$ otherwise.
        \item DFS from unmatched vertices in $X$.
        \item $x \in X$ is chosen iff $x$ is unvisited.
        \item $y \in Y$ is chosen iff $y$ is visited.
    \end{enumerate}
    % \item Minimum cost cyclic flow
    % \vspace{-0.5em}
    % \begin{enumerate}
    %     \itemsep-0.3em
    %     \item Consruct super source $S$ and sink $T$
    %     \item For each edge $(x, y, c)$, connect $x \rightarrow y$ with $(cost, cap) = (c, 1)$ if $c > 0$, otherwise connect $y \rightarrow x$ with $(cost, cap) = (-c, 1)$
    %     \item For each edge with $c < 0$, sum these cost as $K$, then increase $d(y)$ by 1, decrease $d(x)$ by 1
    %     \item For each vertex $v$ with $d(v) > 0$, connect $S \rightarrow v$ with $(cost, cap) = (0, d(v))$
    %     \item For each vertex $v$ with $d(v) < 0$, connect $v \rightarrow T$ with $(cost, cap) = (0, -d(v))$
    %     \item Flow from $S$ to $T$, the answer is the cost of the flow $C + K$
    % \end{enumerate}
    \item Maximum density induced subgraph
    \vspace{-1em}
    \begin{enumerate}
        \itemsep-0.3em
        \item Binary search on answer, suppose we're checking answer $T$
        \item Construct a max flow model, let $K$ be the sum of all weights
        \item Connect source $s \rightarrow v$, $v \in G$ with capacity $K$
        \item For each edge $(u, v, w)$ in $G$, connect $u \rightarrow v$ and $v \rightarrow u$ with capacity $w$
        \item For $v \in G$, connect it with sink $v \rightarrow t$ with capacity $K + 2T - (\sum_{e \in E(v)}{w(e)}) - 2w(v)$
        \item $T$ is a valid answer if the maximum flow $f < K \lvert V \rvert$
    \end{enumerate}
    \item Minimum weight edge cover
    \vspace{-1em}
    \begin{enumerate}
        \itemsep-0.3em
      \item For each $v \in V$ create a copy $v^\prime$, and connect $u^\prime \to v^\prime$ with weight $w(u, v)$.
      \item Connect $v \to v^\prime$ with weight $2\mu(v)$, where $\mu(v)$ is the cost of the cheapest edge incident to $v$.
      \item Find the minimum weight perfect matching on $G^\prime$.
    \end{enumerate}
    \item Project selection problem
    \vspace{-1em}
    \begin{enumerate}
      \itemsep-0.3em
      \item If $p_v > 0$, create edge $(s, v)$ with capacity $p_v$; otherwise, create edge $(v, t)$ with capacity $-p_v$.
      \item Create edge $(u, v)$ with capacity $w$ with $w$ being the cost of choosing $u$ without choosing $v$.
      \item The mincut is equivalent to the maximum profit of a subset of projects.
    \end{enumerate}
    \item 0/1 quadratic programming
    \vspace{-1em}
    \[ \sum_x{c_xx} + \sum_y{c_y\bar{y}} + \sum_{xy}c_{xy}x\bar{y} + \sum_{xyx^\prime y^\prime}c_{xyx^\prime y^\prime}(x\bar{y} + x^\prime\bar{y^\prime}) \]
    can be minimized by the mincut of the following graph:
    \begin{enumerate}
      \itemsep-0.3em
      \item Create edge $(x, t)$ with capacity $c_x$ and create edge $(s, y)$ with capacity $c_y$.
      \item Create edge $(x, y)$ with capacity $c_{xy}$.
      \item Create edge $(x, y)$ and edge $(x^\prime, y^\prime)$ with capacity $c_{xyx^\prime y^\prime}$.
    \end{enumerate}
\end{itemize}

\section{String}
\inputcode{AC Automaton}{String/AhoCorasick.cpp}
\inputcode{Lyndon Factorization}{String/LyndonFactorization.cpp}
\inputcode{KMP}{String/KMP.cpp}
\inputcode{Manacher}{String/Manacher.cpp}
\inputcode{Minimum Rotate}{String/MinRotate.cpp}
\inputcode{Palindrome Tree}{String/PalindromeTree.cpp}
\inputcode{Repetition}{String/Repetition.cpp}
\inputcode{Suffix Array}{String/SA.cpp}
\inputcode{SAIS (C++20)}{String/SAIS_20.cpp}
\inputcode{Suffix Automaton}{String/SAM.cpp}
% \inputcode{exSAM}{String/exSAM.cpp}
% \inputcode{Trie}{String/Trie.cpp}
\inputcode{Z Value}{String/ZValue.cpp}

\section{Math}
\inputcode{Berlekamp Massey}{Math/BerlekampMassey.cpp}
%\inputcode{Combinatorics}{Math/Comb.cpp}
\inputcode{Characteristic Polynomial}{Math/CharPoly.cpp}
%\inputcode{Determinant}{Math/Determinant.cpp}
\inputcode{Discrete Logarithm}{Math/DiscreteLog.cpp}
\inputcode{Extgcd}{Math/Extgcd.cpp}
%\inputcode{Fastpow}{Math/Fastpow.cpp}
\inputcode{Floor Sum}{Math/FloorSum.cpp}
\inputcode{Factorial Mod $P^k$}{Math/FacNoP.cpp}
\inputcode{Gaussian Elimination}{Math/GaussElimination.cpp}
\inputcode{Linear Function Mod Min}{Math/LinearFuncModMin.cpp}
%\inputcode{Matrix}{Math/Matrix.cpp}
\inputcode{MillerRabin PollardRho}{Math/MillerRabinPollardRho.cpp}
%\inputcode{Phi}{Math/Phi.cpp}
\inputcode{Quadratic Residue}{Math/QuadraticResidue.cpp}
%\inputcode{Sieve (With Mu)}{Math/SieveWithMu.cpp}
\inputcode{Simplex}{Math/Simplex.cpp}
\inputcode{FFT}{Math/FFT.cpp}
\inputcode{NTT}{Math/NTT.cpp}
\inputcode{FWT}{Math/FWT.cpp}
\inputcode{Polynomial}{Math/Polynomial.cpp}
\subsection{Generating Functions}
\begin{itemize}
\item Ordinary Generating Function
$A(x) = \sum_{i\ge 0} a_ix^i$
\begin{itemize}
    \itemsep-0.5em
    \item $A(rx)             \Rightarrow r^na_n$
    \item $A(x) + B(x)       \Rightarrow a_n + b_n$
    \item $A(x)B(x)          \Rightarrow \sum_{i=0}^{n} a_ib_{n-i}$
    \item $A(x)^k            \Rightarrow \sum_{i_1+i_2+\cdots+i_k=n} a_{i_1}a_{i_2}\ldots a_{i_k}$
    \item $xA(x)'            \Rightarrow na_n$
    \item $\frac{A(x)}{1-x}  \Rightarrow \sum_{i=0}^{n} a_i$
\end{itemize}
\item Exponential Generating Function
$A(x) = \sum_{i\ge 0} \frac{a_i}{i!}x_i$
\begin{itemize}
    \itemsep-0.5em
    \item $A(x) + B(x)       \Rightarrow a_n + b_n$
    \item $A^{(k)}(x)        \Rightarrow a_{n+k}$
    \item $A(x)B(x)          \Rightarrow \sum_{i=0}^{n} \binom{n}{i}a_ib_{n-i}$
    \item $A(x)^k            \Rightarrow \sum_{i_1+i_2+\cdots+i_k=n} \binom{n}{i_1, i_2, \ldots, i_k}a_{i_1}a_{i_2}\ldots a_{i_k}$
    \item $xA(x)             \Rightarrow na_n$
\end{itemize}
\item Special Generating Function
\begin{itemize}
    \itemsep-0.5em
    \item $(1+x)^n           = \sum_{i\ge 0} \binom{n}{i}x^i$
    \item $\frac{1}{(1-x)^n} = \sum_{i\ge 0} \binom{i}{n-1}x^i$
\end{itemize}
\end{itemize}
\subsection{Linear Programming Construction}
% \normalsize
Standard form: maximize $\mathbf{c}^T\mathbf{x}$ subject to $A\mathbf{x} \leq \mathbf{b}$ and $\mathbf{x} \geq 0$. \\
Dual LP: minimize $\mathbf{b}^T\mathbf{y}$ subject to $A^T\mathbf{y} \geq \mathbf{c}$ and $\mathbf{y} \geq 0$. \\
$\bar{\mathbf{x}}$ and $\bar{\mathbf{y}}$ are optimal if and only if for all $i \in [1, n]$, either $\bar{x}_i = 0$ or $\sum_{j=1}^{m}A_{ji}\bar{y}_j = c_i$ holds and for all $i \in [1, m]$ either $\bar{y}_i = 0$ or $\sum_{j=1}^{n}A_{ij}\bar{x}_j = b_j$ holds.

\begin{enumerate}
    \itemsep-0.5em
    \item In case of minimization, let $c^\prime_i = -c_i$
    \item $\sum_{1 \leq i \leq n}{A_{ji}x_i} \geq b_j \rightarrow \sum_{1 \leq i \leq n}{-A_{ji}x_i} \leq -b_j$
    \item $\sum_{1 \leq i \leq n}{A_{ji}x_i} = b_j$ 
        \vspace{-0.5em}
        \begin{itemize}
            \itemsep-0.5em
            \item $\sum_{1 \leq i \leq n}{A_{ji}x_i} \leq b_j$
            \item $\sum_{1 \leq i \leq n}{A_{ji}x_i} \geq b_j$
        \end{itemize}
    \item If $x_i$ has no lower bound, replace $x_i$ with $x_i - x_i^\prime$
\end{enumerate}
\subsection{Estimation}
% Source: std_abs

\begin{itemize}
    \itemsep-0.5em
    \item The number of divisors of $n$ is at most around $100$ for $n<5e4$, $500$ for $n<1e7$, $2000$ for $n<1e10$, $200000$ for $n<1e19$.
    \item The number of ways of writing $n$ as a sum of positive integers, disregarding the order of the summands. $1, 1, 2, 3, 5, 7, 11, 15, 22, 30$ for $n=0\sim 9$, $627$ for $n=20$, $\sim 2e5$ for $n=50$, $\sim 2e8$ for $n=100$.
    \item Total number of partitions of $n$ distinct elements: $B(n)=1, 1, 2, 5, 15, 52, 203, 877, 4140, 21147, 115975, 678570, 4213597,$\\
    $27644437, 190899322, \ldots$.
\end{itemize}
\subsection{Theorem}
% Source: waynedisonitau123 & 8BQube

\begin{itemize}
\item Kirchhoff's Theorem

Denote $L$ be a $n \times n$ matrix as the Laplacian matrix of graph $G$, where $L_{ii} = d(i)$, $L_{ij} = -c$ where $c$ is the number of edge $(i, j)$ in $G$.
\begin{itemize}
    \item The number of undirected spanning in $G$ is $\lvert \det(\tilde{L}_{11}) \rvert$.
    \item The number of directed spanning tree rooted at $r$ in $G$ is $\lvert \det(\tilde{L}_{rr}) \rvert$.
\end{itemize}

\item Tutte's Matrix

Let $D$ be a $n \times n$ matrix, where $d_{ij} = x_{ij}$ ($x_{ij}$ is chosen uniformly at random) if $i < j$ and $(i, j) \in E$, otherwise $d_{ij} = -d_{ji}$. $\frac{rank(D)}{2}$ is the maximum matching on $G$.

\item Cayley's Formula

\begin{itemize}
  \item Given a degree sequence $d_1, d_2, \ldots, d_n$ for each \textit{labeled} vertices, there are $$\frac{(n - 2)!}{(d_1 - 1)!(d_2 - 1)!\cdots(d_n - 1)!}$$ spanning trees.
  \item Let $T_{n, k}$ be the number of \textit{labeled} forests on $n$ vertices with $k$ components, such that vertex $1, 2, \ldots, k$ belong to different components. Then $T_{n, k} = kn^{n - k - 1}$.
\end{itemize}

\item Erdős–Gallai Theorem

A sequence of non-negative integers $d_1 \geq d_2 \geq \ldots \geq d_n$ can be represented as the degree sequence of a finite simple graph on $n$ vertices if and only if $d_1 + d_2 + \ldots + d_n$ is even and
$$ \sum_{i = 1}^{k}d_i \leq k(k - 1) + \sum_{i = k + 1}^{n}\min(d_i, k) $$
holds for all $1 \leq k \leq n$.

\item Burnside's Lemma

Let $X$ be a set and $G$ be a group that acts on $X$.
For $g \in G$, denote by $X^g$ the elements fixed by $g$:
\[
X^g = \{ x \in X \mid gx \in X \}
\]
Then
\[
|X/G| = \frac{1}{|G|} \sum_{g \in G} |X^g|.
\]

\item Gale–Ryser theorem

A pair of sequences of nonnegative integers $a_1\ge\cdots\ge a_n$ and $b_1,\ldots,b_n$ is bigraphic if and only if $\displaystyle\sum_{i=1}^n a_i=\displaystyle\sum_{i=1}^n b_i$ and $\displaystyle\sum_{i=1}^k a_i\le \displaystyle\sum_{i=1}^n\min(b_i,k)$ holds for every $1\le k\le n$.

\item Fulkerson–Chen–Anstee theorem

A sequence $(a_1,b_1),\ldots,(a_n,b_n)$ of nonnegative integer pairs with $a_1\ge\cdots\ge a_n$ is digraphic if and only if $\displaystyle\sum_{i=1}^n a_i=\displaystyle\sum_{i=1}^n b_i$ and $\displaystyle\sum_{i=1}^k a_i\le \displaystyle\sum_{i=1}^k\min(b_i,k-1)+\displaystyle\sum_{i=k+1}^n\min(b_i,k)$ holds for every $1\le k\le n$.

\item Möbius inversion formula

\begin{itemize}
    \itemsep-0.5em
  \item $f(n)=\sum_{d\mid n}g(d)\Leftrightarrow g(n)=\sum_{d\mid n}\mu(d)f(\frac{n}{d})$
  \item $f(n)=\sum_{n\mid d}g(d)\Leftrightarrow g(n)=\sum_{n\mid d}\mu(\frac{d}{n})f(d)$
\end{itemize}

\item Spherical cap

\begin{itemize}
    \itemsep-0.5em
  \item A portion of a sphere cut off by a plane.
  \item $r$: sphere radius, $a$: radius of the base of the cap, $h$: height of the cap, $\theta$: $\arcsin(a/r)$.
  \item Volume $=\pi h^2(3r-h)/3=\pi h(3a^2+h^2)/6=\pi r^3(2+\cos\theta)(1-\cos\theta)^2/3$.
  \item Area $=2\pi rh=\pi(a^2+h^2)=2\pi r^2(1-\cos\theta)$.
\end{itemize}

\item Chinese Remainder Theorem

\begin{itemize}
  \item $x \equiv a_i \pmod {m_i}$
  \item $M = \prod m_i, M_i = M / m_i$
  \item $t_iM_i \equiv 1 \pmod {m_i}$
  \item $x = \sum a_it_iM_i \pmod M$
\end{itemize}
\end{itemize}
\subsection{General Purpose Numbers}
\begin{itemize}
\item Bernoulli numbers

$B_0=1,B_1^{\pm}=\pm\frac{1}{2},B_2=\frac{1}{6},B_3=0$

$\displaystyle\sum_{j=0}^m\binom{m+1}{j}B_j=0$, EGF is $B(x) = \frac{x}{e^x - 1}=\displaystyle\sum_{n=0}^\infty B_n\frac{x^n}{n!}$.

$S_m(n)=\displaystyle\sum_{k=1}^nk^m=\frac{1}{m+1}\sum_{k=0}^m\binom{m+1}{k}B^{+}_kn^{m+1-k}$

\item Stirling numbers of the second kind
Partitions of $n$ distinct elements into exactly $k$ groups. 

$S(n, k) = S(n - 1, k - 1) + kS(n - 1, k), S(n, 1) = S(n, n) = 1$

$S(n, k) = \frac{1}{k!}\sum_{i=0}^{k}(-1)^{k-i}{k \choose i}i^n$

$x^n     = \sum_{i=0}^{n} S(n, i) (x)_i$

\item Pentagonal number theorem

$\displaystyle\prod_{n=1}^{\infty}(1-x^n)=1+\sum_{k=1}^{\infty}(-1)^k\left(x^{k(3k+1)/2} + x^{k(3k-1)/2}\right)$

\item Catalan numbers

$C^{(k)}_n = \displaystyle \frac{1}{(k - 1)n + 1}\binom{kn}{n}$

$C^{(k)}(x) = 1 + x [C^{(k)}(x)]^k$

\item Eulerian numbers

Number of permutations $\pi \in S_n$ in which exactly $k$ elements are greater than the previous element. $k$ $j$:s s.t. $\pi(j)>\pi(j+1)$, $k+1$ $j$:s s.t. $\pi(j)\geq j$, $k$ $j$:s s.t. $\pi(j)>j$.

$E(n,k) = (n-k)E(n-1,k-1) + (k+1)E(n-1,k)$

$E(n,0) = E(n,n-1) = 1$

$E(n,k) = \sum_{j=0}^k(-1)^j\binom{n+1}{j}(k+1-j)^n$

\end{itemize}


\section{Geometry}
\inputcode{Basic}{Geometry/Basic.cpp}
\inputcode{Convex Hull}{Geometry/ConvexHull.cpp}
\inputcode{Dynamic Convex Hull}{Geometry/DynConvexHull.cpp}
\inputcode{Point In Convex Hull}{Geometry/PointInConvex.cpp}
\inputcode{Point In Circle}{Geometry/PointInCircle.cpp}
\inputcode{Half Plane Intersection}{Geometry/HalfPlane.cpp}
\inputcode{Minkowski Sum}{Geometry/Minkowski.cpp}
\inputcode{Polar Angle}{Geometry/PolarAngle.cpp}
\inputcode{Rotating Sweep Line}{Geometry/RotatingSweep.cpp}
\inputcode{Segment Intersect}{Geometry/SegIntersect.cpp}
\inputcode{Circle Intersect With Any}{Geometry/CircleIntersect.cpp}
\inputcode{Tangents}{Geometry/Tangent.cpp}
\inputcode{Tangent to Convex Hull}{Geometry/TangentPointToHull.cpp}
\inputcode{Minimum Enclosing Circle}{Geometry/MinEncloseCircle.cpp}
\inputcode{Union of Stuff}{Geometry/Union.cpp}
\inputcode{Delaunay Triangulation}{Geometry/DelaunayTriangulation_dq.cpp}
\inputcode{Voronoi Diagram}{Geometry/Voronoi.cpp}
\inputcode{Trapezoidalization}{Geometry/Trapezoidalization.cpp}
\inputcode{3D Basic}{Geometry/Basic3D.cpp}
\inputcode{3D Convex Hull}{Geometry/ConvexHull3D.cpp}

\section{Misc}
\inputcode{Binary Search On Fraction}{Misc/BinarySearchOnFraction.cpp}
\inputcode{Cyclic Ternary Search}{Misc/cyc_tsearch.cpp}
\inputcode{Min Plus Convolution}{Misc/MinPlusConv.cpp}
\inputcode{Mo's Algorithm}{Misc/MoAlgorithm.cpp}
\inputcode{Mo's Algorithm On Tree}{Misc/MoAlgorithmOnTree.cpp}
\inputcode{PBDS}{Misc/PBDS.cpp}
%\inputcode{Pragma}{Misc/Pragma.cpp}
\inputcode{Simulated Annealing}{Misc/SimulatedAnnealing.cpp}
\inputcode{SOS dp}{Misc/SOS.cpp}
\inputcode{SMAWK}{Misc/SMAWK.cpp}
\inputcode{Tree Hash}{Misc/TreeHash.cpp}
% 3D Partial Order: tested with Library Checker Rectangle Add Point Get, Luogu P3810
\inputcode{3D Partial Order}{Misc/3dPartialOrder.cpp}
\inputcode{Python}{Misc/python.py}


\end{document}
